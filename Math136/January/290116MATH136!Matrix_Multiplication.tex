\documentclass{letter}
\usepackage[margin=0.75in]{geometry}
\usepackage{amsmath}
\usepackage{amssymb}
\usepackage{enumerate}
\usepackage{changepage}
\usepackage{tikz}
\usepackage{pgfplots}
\pgfplotsset{compat=1.8}

\pgfplotsset{vasymptote/.style={
		before end axis/.append code={
			\draw[densely dashed] ({rel axis cs:0,0} -| {axis cs:#1,0})
			-- ({rel axis cs:0,1} -| {axis cs:#1,0});
		}
	}}
	
\newcommand{\m}{\begin{bmatrix}}
\newcommand{\mm}{\end{bmatrix}}
\newcommand{\0}[1]{\begin{bmatrix}#1\end{bmatrix}}
\newcommand{\h}[1]{\underline{\textbf{#1}}}	

\begin{document}
	\begin{center}
		\LARGE Math136 - January 29'th, 2016\\
		\large Matrix Multiplication
	\end{center}
	\vspace{0.25 in}
	
	\h{Transpose}
	
	The \textbf{transpose} of a $m \times n$ matrix $A$ is the $n \times m$ matrix $A^T$ where the $ij$th entry is the $ji$th entry of $A$. That is:\\
	$(A^T)_{ij} = (A)_{ji}$
	
	\h{Theorem 3.1.2} If $A, B \in M_{m\times n} (\mathbb{R})$ then:
	\begin{enumerate}[1)]
		\item $(A^T)^T = A$
		\item $(A+B)^T = A^T + B^T$
		\item $(cA)^T = cA^T$
	\end{enumerate}
	
	Note: We will often view vectors in $\mathbb{R}^n$ as $n \time 1$ matrices.
	
	\h{Matrix - Vector Multiplication}
	
	Suppose $A$ is a $m \times n$ matrix with rows $(\vec a_1)^T, (\vec a_2)^T, \dots, (\vec a_m)^T$ for $\vec a_i \in \mathbb{R}^n$. Then for $\vec x \in \mathbb{R}^n$, we define $A\vec x$ by:
	
	$A\vec x = \0{\vec a_1 \cdot \vec x\\\vdots\\\vec a_m \cdot \vec x}$
	
	And this defines the multiplication of a $m \times n$ matrix by a vector (or equivalently, an $n \times 1$ matrix)
	
	The result is an $m \times 1$ matrix.
	
	So ($m \times n$ matrix) times ($n \times 1$ matrix) gives ($ \times 1$ matrix)
	
	\h{Matrix - Matrix Multiplication}
	
	Suppose $A$ is an $m \times n$ matrix, and $B$ is a $n \times p$ matrix with columns $\vec b_1, \vec b_2, \dots, \vec b_p$
	
	Then $AB = A\0{\vec b_1&\vec b_2&\dots & \vec b_p} = \0{A\vec b_1&A\vec b_2&\dots&A\vec b_p}$
	
	\h{Matrix Manipulation Theorem}
	
	If $A, B, C$ are matrices so that all following products are defined, we have:
	\begin{enumerate}[1)]
		\item $A(B+C) = AB + BC$
		\item $t(A+B) = (tA)B$
		\item $A(BC) = (AB)C$
		\item $(AB)^T = B^TA^T$
	\end{enumerate}
\end{document}