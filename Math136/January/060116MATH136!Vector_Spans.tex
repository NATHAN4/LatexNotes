\documentclass{letter}
\usepackage[margin=0.75in]{geometry}
\usepackage{amsmath}
\usepackage{amssymb}
\usepackage{enumerate}
\usepackage{changepage}
\usepackage{tikz}
\usepackage{pgfplots}
\pgfplotsset{compat=1.8}

\pgfplotsset{vasymptote/.style={
		before end axis/.append code={
			\draw[densely dashed] ({rel axis cs:0,0} -| {axis cs:#1,0})
			-- ({rel axis cs:0,1} -| {axis cs:#1,0});
		}
	}}

\begin{document}
	\begin{center}
		\LARGE Math136 - January 6'th, 2016\\
		\large Vector Spans
	\end{center}
	\vspace{0.25 in}
	
	\underline{\textbf{Proof}}
	
	We will prove Theorem 1 Property 4.
	
	\textbf{Recall Property 4: } There exists $\vec 0 \in \mathbb{R}^n$ such that $\vec x + \vec 0 = \vec 0 + \vec x = \vec x$ for any $\vec x \in \mathbb{R}^n$
	
	Because this is a 'there exists' proof, we will show that the vector $\begin{bmatrix}0\\0\\\vdots\\0\end{bmatrix} \in \mathbb{R}^n$ has these properties.
	
	Now for $\vec x = \begin{bmatrix}x_1\\x_2\\\vdots\\x_n\end{bmatrix} \in \mathbb{R}^n$, we have $\vec x + \vec 0 = \begin{bmatrix}x_1\\x_2\\\vdots\\x_n\end{bmatrix} + \begin{bmatrix}0\\0\\\vdots\\0\end{bmatrix} = \begin{bmatrix}x_1 + 0\\x_2 + 0\\\vdots\\x_n + 0\end{bmatrix} = \begin{bmatrix} x_1\\x_2\\\vdots\\x_n\end{bmatrix}$ (By the definition of vector additions).
	
	Similarly, $\vec 0 + \vec x = \vec x \; \square$
	
	\underline{\textbf{Span}}
	
	For a set $B = \{ \vec v_1, \vec v_2, \dots, \vec v_k \}$ of vectors from $\mathbb{R}^n$, span($B$) = $\{ c_1 \vec v_1 + \dots + c_k \vec v_k \mid c_1, \dots, c_k \in \mathbb{R} \}$ is the set of all linear combinations of the vectors in $B$.
	
	E.g. Give a geometric description of span($\{ \begin{bmatrix}2\\1\end{bmatrix} \}$)\\
	$= \{ c_1\begin{bmatrix}2\\1\end{bmatrix} \mid \in \mathbb{R} \}$
	
	If we look at this vector graphically on the 2D plane, its a line from the origin to (2, 1). The span is the set of all vectors that are a scalar multiple of the original vector(s). In this example, it means the span of this vector is any point on the line shown below: (OH BOY GRAPH TIME)
	\begin{center}
	\begin{tikzpicture}
	\begin{axis}[
	axis equal image,
	axis lines=middle,
	xmin=-3,xmax=3,
	ymin=-3,ymax=3,
	enlargelimits={abs=1cm},
	axis line style={latex-latex},
	yticklabel style={anchor=west},
	ytick={1},
	xtick={2},
	]
	% This doesn't clip to y=-10:10 nicely
	% because there are too few samples near the asymptote:
	\addplot[very thick, black, domain=-10:10,samples=200, restrict y to domain=-100:100]
	{((0.5) * x};
	\addplot[only marks] table {
		2 1	
	};
	
	
	\end{axis}
	\end{tikzpicture}
	\end{center}
	E.g. Describe span($\{ \begin{bmatrix}1\\0\\0\end{bmatrix}, \begin{bmatrix}0\\0\\1\end{bmatrix} \}$) in $\mathbb{R}^3$)\\
	$= \{ c_1 \begin{bmatrix}1\\0\\0\end{bmatrix}+ c_2 \begin{bmatrix}0\\0\\1\end{bmatrix} \mid c_1, c_2 \in \mathbb{R} \}$ 
	$= \{ \begin{bmatrix}c_1\\0\\c_2\end{bmatrix} \}$ (By expanding and adding)\\
	This is the set of vectors $\vec x \in \mathbb{R}^3$ such that $x_2 = 0$ (The $x_1x_3$ plane).
	\clearpage
	
	E.g. Consider: span($\{ \begin{bmatrix}1\\1\\1\end{bmatrix}, \begin{bmatrix}2\\1\\3\end{bmatrix}, \begin{bmatrix}0\\-1\\1\end{bmatrix} \}$)
	
	Note that $\begin{bmatrix}2\\1\\3\end{bmatrix} = 2\begin{bmatrix}1\\1\\1\end{bmatrix} + \begin{bmatrix}0\\-1\\1\end{bmatrix}$
	
	As we will soon see, it follows that span($\{ \begin{bmatrix}1\\1\\1\end{bmatrix}, \begin{bmatrix}2\\1\\3\end{bmatrix}, \begin{bmatrix}0\\-1\\1\end{bmatrix}\}$) = span($\{ \begin{bmatrix}1\\1\\1\end{bmatrix}, \begin{bmatrix}0\\-1\\1\end{bmatrix} \}$)
	
	We will prove this soon. For now, we'll do an illustrative example.
	
	Consider the vector $\vec v = 2\begin{bmatrix}1\\1\\1\end{bmatrix} - 3\begin{bmatrix}2\\1\\3\end{bmatrix} + \begin{bmatrix}0\\-1\\1\end{bmatrix}$ in $ \mathbb{R}^3$
	
	Lets show that $\vec v \in $ span($ \{ \begin{bmatrix}1\\1\\1\end{bmatrix}, \begin{bmatrix}0\\-1\\1\end{bmatrix} \} $)
	\begin{flalign*}
		\vec v &= 2\begin{bmatrix}1\\1\\1\end{bmatrix} - 3(2\begin{bmatrix}1\\1\\1\end{bmatrix} + \begin{bmatrix}0\\-1\\1\end{bmatrix}) + \begin{bmatrix}0\\-1\\1\end{bmatrix}&\\
		&= 2\begin{bmatrix}1\\1\\1\end{bmatrix} -6\begin{bmatrix} 1\\1\\1\end{bmatrix} - 3\begin{bmatrix}0\\-1\\1\end{bmatrix} + \begin{bmatrix}0\\-1\\1\end{bmatrix}\\
		&= -4\begin{bmatrix}1\\1\\1\end{bmatrix} - 2\begin{bmatrix}0\\-1\\1\end{bmatrix}
	\end{flalign*}
	
	\underline{\textbf{Theorem 2}}
	
	Let $\vec v_1, \vec v_2, \dots, \vec v_k \in \mathbb{R}^n$. Then, for an index $i \in \{ 1, \dots, k \}$ we have span($\{ \vec v_1, \dots, \vec v_k \}$) = span($  \{ \vec v_1,\dots, \vec v_{i-1}, \vec v_{i+1}, \dots, \vec v_k \} $) iff $\vec v_i$ can be written as a linear combinations of $\vec v_1, \dots, \vec v_{i-1}, \vec v_{i+1}, \dots, \vec v_k$
	
	Proof:
	
	\begin{itemize}
		\item[$\impliedby$] Suppose $\vec v_i = c_1 \vec v_1 + \dots + c_{i-1}\vec v_{i-1} + c_{i+1} \vec v_{i+1} + \dots + c_k \vec v_k$\\\\
		Then, suppose $\vec v = d_1 \vec v_1 + \dots + d_k \vec v_k \in $ span($\{ \vec v_1, \dots, \vec v_k \}$)\\
		\begin{flalign*}
			\text{Then, } \vec v &= d_1 \vec v_1 + \dots + d_{i-1} \vec v_{i-1} + d_i \vec v_i + d_{i+1} \vec v_{i+1} + \dots + d_k + \vec v_k&\\
			&= d_1 \vec v_1 + \dots + d_{i-1} \vec v_{i-1} + d_{i+1} \vec v_{i+1} + \dots + d_k \vec v_k + d_i(c_1 \vec v_1 + \dots + c_{i-1} \vec v_{i-1} + c_{i+1} \vec v_{i+1} + \dots + c_k \vec v_k)\\
			&\;\;\text{Distribute and Group!}\\
			&= (d_1 + d_ic_1)\vec v_1 + \dots + (d_{i-1} + d_i c_{i-1})\vec v_{i-1} + (d_{1+1} + d_ic_{i+1})\vec v_{i+1} + \dots + (d_k + d_ic_k) \vec v_k \in \\
			&\;\;\;\;\text{ span}(\{ \vec v_1, \dots, \vec v_{i-1}, \vec v_{i+1}, \dots, \vec v_k \})
		\end{flalign*}
		
		So, we have showed the left hand side of the equality is a subset of the right hand side.\\
		Now we will show the right hand side is a subset of the left hand side.
		\begin{flalign*}
			\vec v &= c_1 \vec v_1 + \dots + c_{i-1} \vec v_{i-1} + c_{i+1} \vec v_{i+1} + \dots + v_k \vec v_k&\\
			&= c_1 \vec v_1 + \dots + c_{i-1} \vec v_{i-1} + (0)\vec v_i + c_{i+1} \vec v_{i+1} + \dots + v_k \vec v_k
		\end{flalign*}
		
		TO BE CONTINUED!!
	\end{itemize}
\end{document}