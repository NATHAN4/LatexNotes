\documentclass{letter}
\usepackage[margin=0.75in]{geometry}
\usepackage{amsmath}
\usepackage{amssymb}
\usepackage{enumerate}
\usepackage{changepage}
\usepackage{tikz}
\usepackage{pgfplots}
\pgfplotsset{compat=1.8}

\pgfplotsset{vasymptote/.style={
		before end axis/.append code={
			\draw[densely dashed] ({rel axis cs:0,0} -| {axis cs:#1,0})
			-- ({rel axis cs:0,1} -| {axis cs:#1,0});
		}
	}}
	
\newcommand{\m}{\begin{bmatrix}}
\newcommand{\mm}{\end{bmatrix}}
\newcommand{\0}[1]{\begin{bmatrix}#1\end{bmatrix}}
\newcommand{\h}[1]{\underline{\textbf{#1}}}	

\begin{document}
	\begin{center}
		\LARGE Math136 - January 27'th, 2016\\
		\large Rank, Solution Theorem, Matrices
	\end{center}
	\vspace{0.25 in}
	
	\h{Rank of a Matrix}
	
	Suppose $A$ is a matrix with RREF $R$. The \textbf{rank} of $A$ is the number of leading ones in $R$.
	
	\h{Theorem 2.2.4}
	
	If $A$ is a $m\times n$ matrix, then Rank $A \leq \min(m, n)$.
	
	\h{Theorem 2.2.5}
	
	Let $A$ be the coefficient matrix of a system of $m$ linear equations in $n$ unknowns with augmented matrix $\0{A&\vert&\vec b}$.
	
	\begin{enumerate}
		\item If the rank of $A$ is less than the rank of the augmented matrix $\0{A&\vert&\vec b}$, then the system is consistent.
		
		\item If the system $\0{A&\vert&\vec b}$ is consistent, then the system contains ($n - $rank A) free variables.
		
		\item Rank $A$ = $m$ iff the system is consistent for every $\vec b \in \mathbb{R}^m$
	\end{enumerate}
	
	This is the \textbf{System-Rank Theorem}
	
	Note: Suppose that $A$ is the coefficient matrix or a system of equations with augmented matrix $\0{A&\vert&\vec b}$. Then if $A$ has RREF $R$, then $\0{A&\vert&\vec b}$ will have RREF $\0{R&\vert&\vec r}$ for some $\vec r$
	
	\h{Theorem 2.2.6}
	
	Let $\0{A&\vert&\vec b}$ be a consistent system of $m$ linear equations in $n$ variables with RREF $\0{R&\vert&\vec b}$. If rant $A = k < n$, then a vector equation of the solution set of $\0{A&\vert&\vec b}$ has the form $\vec x = \vec d + t_1\vec v_1 + \dots + t_{n-k}\vec v_{n-k}, t_1, \dots, t_{n-k} \in \mathbb{R}$ where $\vec d \in \mathbb{R}^n$ and $\{ \vec v_1, \dots, \vec v_{n-k} \}$ is a linearly independent set of vectors in $\mathbb{R}^n$. So the solution set is an ($n-k$)-flat in $\mathbb{R}^n$
	
	A system of linear equations is \textbf{homogeneous} if the right hand side is zero. The set of solutions to a homogeneous system is the solution space.
	
	Note: A homogeneous system is always consistent since we get a solution by setting all vars = 0.
	
	\h{Matrices}
	
	A $m\times n$ matrix $A$ is a rectangular array with $m$ rows and $n$ columns. We denote the entry in the $i$th row and $j$th column by $a_{ij}$, that is:
	
	$A = \0{a_{11}&a_{12}&\dots&a_{1n}\\a_{21}&a_{22}\\\vdots&&\ddots\\a_{m1}&&&a_{mn}}$
	
	\h{Addition and Scalar Multiplication}
	
	Let $A, B \in M_{m\times n} (\mathbb{R})$ and $c \in \mathbb{R}$. We define $A + B$ and $cA$ as:
	
	$(A+B)_{ij} = A_{ij} + B_{ij}$
	
	$(cA)_{ij} = c(A)_{ij}$
	
	\h{Theorem 3.1.1}
	
	The set of matrices in $M_{m\times n} (\mathbb{R})$ satisfy the properties analogous to those in theorem 1.1.1.
\end{document}