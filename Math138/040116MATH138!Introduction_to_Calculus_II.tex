\documentclass{letter}
\usepackage[margin=0.75in]{geometry}
\usepackage{amsmath}
\usepackage{amssymb}
\usepackage{enumerate}
\usepackage{changepage}
\usepackage{tikz}
\usepackage{pgfplots}
\pgfplotsset{compat=1.8}

\pgfplotsset{vasymptote/.style={
		before end axis/.append code={
			\draw[densely dashed] ({rel axis cs:0,0} -| {axis cs:#1,0})
			-- ({rel axis cs:0,1} -| {axis cs:#1,0});
		}
	}}

\begin{document}
	\begin{center}
		\LARGE Math138 - January 4'th, 2016\\
		\large Introduction to Calculus II
	\end{center}
	\vspace{0.25 in}
	
	\underline{\textbf{Administrative Stuff}}
	
	\begin{minipage}[t]{0.2\textwidth}
		Professor:\\
		Office:\\
		Office Hours:\\
		Email:\\
		Assignments Due:\\
		First Assignment Due:\\
	\end{minipage}
	\begin{minipage}[t]{0.8\textwidth}
		Jordan Hamilton\\
		MC 6502\\
		4-6PM Mondays, Wednesdays, MC5417\\
		j4hamilt@uwaterloo.ca\\
		Fridays at 2PM\\
		January 15'th
	\end{minipage}
	
	\underline{\textbf{This Week:}}
	\begin{itemize}
		\item Review Integration (Chapter 5)
		\item Integration by Parts (Chapter 7.1)
		\item Trig Integrals (Chapter 7.2)
	\end{itemize}
	
	\underline{\textbf{Integration}}
	
	We already know lots of integrals (anti-derivatives). Ones we should know include powers of $x$, $\frac{1}{x}$, $e^x$, basic trig, trig inverses and hyperbolic trig functions.
	
	\begin{minipage}[t]{0.1\textwidth}
		Recall:\\
	\end{minipage}
	\begin{minipage}[t]{0.8\textwidth}
		Definite integrals represent the area below the curve from $x=a$ to $x=b$, shown as $\displaystyle \int_a^b f(x) dx$\\
		We use the Fundamental Theorem of Calculus (FTC) to evaluate.
	\end{minipage}
	
	\underline{\textbf{Fundamental Theorem of Calculus}}
	
	If $f(x)$ is continuous and...$\;\;\;\;$
	\begin{minipage}[t]{0.2\textwidth}
		Part 1:\\
		Part 2:
	\end{minipage}
	\begin{minipage}[t]{0.8\textwidth}
		...$F(x) = \int_a^x f(t) dt$ then $F'(x) = f(x)$\\
		...If $f(x)$ is any anti-derivative of $f(x)$ (so $F'(x) = f(x)$), then:\\
		$\displaystyle \int_a^b f(x) dx = F(b) - F(a)$
	\end{minipage}
	
	\begin{itemize}
		\item[Ex. ] $\displaystyle \int_0^1 x^2 dx$\\
		$\displaystyle =\left[ \frac{x^3}{3}\right]_0^1$\\
		$\displaystyle =\frac{1^3}{3}$\\
		$\displaystyle = \frac{1}{3}$
	\end{itemize}
	
	We can also use FTC tp find the derivative of integral functions.\\
	Suppose $P(x) = \int_{g(x)}^{h(x)}f(t) dt$\\
	We want to solve for $P'(t)$
	\begin{itemize}
		\item[\;\;] Let $F(x)$ be an anti-derivative of $f(x)$. Then by FTC, $P(x) = F(h(x)) - F(g(x))$
		\begin{flalign*}
			P'(x) &= F'(h(x)) \cdot h'(x) - F'(g(x)) \cdot g'(x)&\\
			&= f(h(x)) \cdot h'(x) - f(g(x)) \cdot g'(x)
		\end{flalign*}
	\end{itemize}
	
	\clearpage
	
	\underline{\textbf{U-Substitution (Reverse Chain Rule)}}
	
	$\int f(g(x)) \cdot g'(x) dx = \int f(u) du$ where $u = g(x)$
	
	But how do we pick $u$?
	\begin{itemize}
		\item If you see a function together with its derivative, let $u$ be this function.
		\item Let $u$ = the base of an ugly power or denominator.
		\item Let $u$ = whats inside a function like sin, log, etc.
	\end{itemize}
	
	\underline{\textbf{U-Sub Examples}}
	
		\begin{minipage}[t]{0.3\textwidth}
			\begin{flalign*}
				\text{a)\;\;\;\;}&\int \frac{\ln x}{x}dx&\\
				= &\int u\; du\\
				= &\frac{u^2}{2} + c
			\end{flalign*}
		\end{minipage}
		\begin{minipage}[t]{0.5\textwidth}
			$\;$\\\textbf{Side Work}\\
			Let $u = \ln x$\\
			$x\;du = dx$
		\end{minipage}
		
		\begin{minipage}[t]{0.3\textwidth}
			\begin{flalign*}
			\text{b)\;\;\;\;}&\int_0^1 \frac{x^3}{1+x^4}\;dx&\\
			&= \int_1^2 \frac{x^3}{u} \cdot \frac{du}{4x^3}\\
			&= \int_1^2 \frac{1}{4u}\;du\\
			&= \frac14 \int_1^2 \frac{1}{u}\;du\\
			&= \frac14 \left[ \ln \vert u \vert \right]_1^2\\
			&= \frac{1}{4}(\ln 2 - \ln 1)\\
			&= \frac{\ln 2}{4}
			\end{flalign*}
		\end{minipage}
		\begin{minipage}[t]{0.5\textwidth}
			$\;$\\\textbf{Side Work}\\
			Let $u = 1 + x^4$\\
			$\frac{du}{dx} = 4x^3$\\
			$\frac{du}{4x^3} = dx$\\
			If $x = 0$, $u = 1$\\
			If $x = 1$, $u = 2$
		\end{minipage}
\end{document}