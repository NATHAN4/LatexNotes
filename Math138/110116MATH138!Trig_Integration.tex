\documentclass{letter}
\usepackage[margin=0.75in]{geometry}
\usepackage{amsmath}
\usepackage{amssymb}
\usepackage{enumerate}
\usepackage{changepage}
\usepackage{tikz}
\usepackage{pgfplots}
\pgfplotsset{compat=1.8}

\pgfplotsset{vasymptote/.style={
		before end axis/.append code={
			\draw[densely dashed] ({rel axis cs:0,0} -| {axis cs:#1,0})
			-- ({rel axis cs:0,1} -| {axis cs:#1,0});
		}
	}}
	
\newcommand{\m}{\begin{bmatrix}}
\newcommand{\mm}{\end{bmatrix}}
\newcommand{\0}[1]{\begin{bmatrix}#1\end{bmatrix}}
\newcommand{\h}[1]{\underline{\textbf{#1}}}	

\begin{document}
	\begin{center}
		\LARGE Math138 - January 11'th 2016\\
		\large Trig Integration
	\end{center}
	\vspace{0.25 in}
	
	\h{Trig Integration - Cases}
	
	Sometimes letting $x$ equal a trig function can simplify an integral.
	
	We'll be looking at three main cases:
	
	\begin{enumerate}[1)]
		\item $\sqrt{a^2 - x^2} \implies x = a\sin \theta$
		\item $\sqrt{a^2 + x^2} \implies x = a\tan \theta$
		\item $\sqrt{x^2 - a^2} \implies x = a\sec \theta$
	\end{enumerate}
	
	\h{Three Warnings}
	
	\begin{enumerate}[1)]
		\item Watch for simpler methods of integration!
		\item You may need to simplify before making a trig sub.
		\item Always switch back to $x$ at the end.
	\end{enumerate}
	
	\h{Examples}
	
	\begin{enumerate}[1)]
		\item \begin{minipage}[t]{0.3\textwidth}
			\begin{flalign*}
			&\int \frac{1}{\sqrt{x^2 + 4}}\;dx&\\
			= &\int \frac{2 \sec^2 \theta}{\sqrt{4 \tan^2 \theta + 4}}\\
			= &\int \frac{\sec^2 \theta}{\sqrt{\tan^2 \theta + 1}}\\
			= &\int \frac{\sec^2 \theta}{\sqrt{\sec^2 \theta}}\\
			= &\int \sec \theta \; d\theta\\
			= &\ln \vert\;\sec \theta + \tan \theta\;\vert\; + c\\
			= &\ln \vert\; \frac{\sqrt{x^2 + 4}}{2} + \frac{x}{2}\;\vert\; + c
			\end{flalign*}
		\end{minipage}
		\begin{minipage}[t]{0.5\textwidth}
			\begin{flalign*}
			x &= 2 \tan \theta&\\
			dx &= 2 \sec^2 \theta \; d\theta\\
			&\text{NOTE: USE TRIANGLES TO GET THESE VALUES}\\
			\tan \theta &= \frac{x}{2}\\
			\sec \theta &= \frac{\sqrt{x^2 + 4}}{2}\\
			\end{flalign*}
		\end{minipage}
		\clearpage
		\item \begin{minipage}[t]{0.3\textwidth}
			\begin{flalign*}
			&\int \frac{\sqrt{9 - 4x^2}}{x^2}\; dx\\
			= &2 \int \frac{\sqrt{\frac{9}{4} - x^2}}{x^2}\\
			= &2 \int \frac{\sqrt{\frac{9}{4} - \frac{9}{4}\sin \theta}}{\frac{9}{4}\sin^2 \theta} \cdot \frac32 \cos \theta \; d\theta\\
			= &2 \cdot \frac32 \cdot \frac32 \cdot \frac49 \int \frac{\sqrt{1-\sin^2 \theta} \cdot \cos \theta}{\sin^2 \theta}\\
			= &2 \int \frac{\cos^2 \theta}{\sin^2 \theta}\\
			= &2 \int \cot^2 \theta\\
			= &2 \int \csc^2 \theta - 1\\
			= &2 (-\cot \theta - \theta) + c\\
			= &2 (\frac{-\sqrt{9 - 4x^2}}{2x} - \arcsin \frac{2x}{3})
			\end{flalign*}
		\end{minipage}
		\begin{minipage}[t]{0.5\textwidth}
			\begin{flalign*}
			x &= \frac{3}{2}\sin \theta&\\
			dx &= \frac32 \cos \theta\; d\theta\\
			\frac{2x}{3} &= \sin \theta\\
			\cot \theta &= \frac{\sqrt{9 - 4x^2}}{2x}
			\end{flalign*}
		\end{minipage}
		\item \begin{minipage}[t]{0.3\textwidth}
			\begin{flalign*}
				&\int \frac{dx}{x^2 \sqrt{x^2 - 4}}&\\
				= &\int \frac{2 \sec \theta \tan \theta}{4 \sec^2 \theta \sqrt{4 \sec^2 \theta - 4}}\\
				= &\frac14 \int \frac{\sec \theta \tan \theta}{\sec^2 \theta \sqrt{\sec^2 \theta - 1}}\\
				= &\frac14 \int \frac{\tan \theta}{\sec \theta}\\
				= &\frac14 \int \cos \theta\\
				= &\frac14 \sin \theta\\
				= &\frac14 \frac{\sqrt{x^2 - 4}}{x}
			\end{flalign*}
		\end{minipage}
		\begin{minipage}[t]{0.5\textwidth}
			\begin{flalign*}
				x &= 2 \sec \theta&\\
				\frac{x}{2} &= \sec \theta\\
				\frac{2}{x} &= \cos \theta\\
			\end{flalign*}
		\end{minipage}
		\clearpage
		\item \begin{minipage}[t]{0.3\textwidth}
			\begin{flalign*}
			&\int x \sqrt{x^2 - 9}\; dx&\\
			= &\frac12 \int \sqrt{u}\; du\\
			= &\frac13 u^{3/2} + c\\
			= &\frac13 (x^2 - 9)^{3/2} + c
			\end{flalign*}
		\end{minipage}
		\begin{minipage}[t]{0.5\textwidth}
			\begin{flalign*}
				u &= x^2 - 9&\\
				du &= 2x\; dx\\
				\frac{du}{2x} = dx
			\end{flalign*}
		\end{minipage}
		\item \begin{minipage}[t]{0.3\textwidth}
			\begin{flalign*}
				&\int \frac{x^2}{(1+x^2)^2}\; dx&\\
				= &\int \frac{\tan^2 \theta \sec^2 \theta}{\sec^4 \theta}\\
				= &\int \frac{\tan^2 \theta}{\sec^2 \theta}\\
				= &\int \left( \frac{\sin^2 \theta}{\cos^2 \theta}\right)\left(\cos^2 \theta\right)\\
				= &\int \sin^2 \theta\\
				= &\frac12 \int 1 - \cos(2\theta)\; d\theta\\
				= &\frac12\left(\theta - \frac{\sin(2\theta)}{2}\right) + c\\
				= &\frac12\left(\arctan x - \frac{x}{1+x^2}\right) + c
			\end{flalign*}
		\end{minipage}
		\begin{minipage}[t]{0.5\textwidth}
			\begin{flalign*}
				x &= \tan \theta&\\
				dx &= \sec^2 \theta\\
				\sin(2\theta) &= 2\sin \theta \cos \theta\\
				&= \frac{2x}{1+x^2}
			\end{flalign*}
		\end{minipage}
	\end{enumerate}
\end{document}