\documentclass{letter}
\usepackage[margin=0.75in]{geometry}
\usepackage{amsmath}
\usepackage{amssymb}
\usepackage{enumerate}
\usepackage{changepage}
\usepackage{tikz}
\usepackage{pgfplots}
\pgfplotsset{compat=1.8}

\pgfplotsset{vasymptote/.style={
		before end axis/.append code={
			\draw[densely dashed] ({rel axis cs:0,0} -| {axis cs:#1,0})
			-- ({rel axis cs:0,1} -| {axis cs:#1,0});
		}
	}}
	
\newcommand{\m}{\begin{bmatrix}}
\newcommand{\mm}{\end{bmatrix}}
\newcommand{\0}[1]{\begin{bmatrix}#1\end{bmatrix}}
\newcommand{\h}[1]{\underline{\textbf{#1}}}	

\begin{document}
	\begin{center}
		\LARGE Math138 - February 1'st, 2016\\
		\large Solving Differentials - Separable ODEs
	\end{center}
	\vspace{0.25 in}
	
	\h{Separable ODEs}
	
	A separable Ordinary Differential Equation is an equation where we can get a function of x on one side and a function of y on the other side. If we can achieve this, we can integrate both sides like normal.
	
	\h{Separable ODE Definition}
	
	A separable ODE is a first-order ODE that can be written as $\frac{dy}{dx} = g(y) \cdot h(x)$. IE, we can separate into a product of functions, one containing only $y$, the other containing only $x$
	
	To solve these, move $g(y)$ to the LHS and integrate.
	
	\begin{flalign*}
		\frac{1}{g(y)} \frac{dy}{dx} &= h(x)&\\
		\int \frac{1}{g(y)} \frac{dy}{dx} dx &= \int h(x)\;dx\\
		\int \frac{1}{g(y)}\; dx &= \int h(x)\; dx
	\end{flalign*}
	
	Note: Normally you cannot treat the differential as a fraction and cancel out $dy$ or $dx$ in this example, we actually use a substitution and skip a step, so we've not broken any rules.
	
	\begin{itemize}
		\item[\textbf{Ex. }] Find the particular solution to the IVP $\frac{dy}{dx} = \frac{3x^2 + 4x + 2}{2(y-1)},\;\;\;\;y(0) = -1$
		
		Notice we can separate this into functions of $x$ and functions of $y$!
		\begin{flalign*}
			2(y-1) dy &= 3x^2 + 4x + 2&\\
			\int 2y - 2\; dy &= \int 3x^2 + 4x + 2\; dx\\
			2y^2 - 2y &= x^3 + 2x^2 + 2x + c\\\\
			(-1)^2 - 2(-1) &= 0 + 0 + 0 + c\\
			-3 &= c\\
			&\text{So, we plug in $c$ and try to solve for y}\\
			y^2 - 2y &= x^3 + 2x^2 + 2x + 3\\
			(y-1)^2 - 1 &= x^2 + 2x^2 + 2x + 3\\
			y-1 &= \pm \sqrt{x^3 + 2x^2 + 2x + 4}\\
			y &= 1 \pm \sqrt{x^3 + 2x^2 + 2x + 4}
		\end{flalign*}
		But only one of these satisfies $y(0) = -1$
		
		So $y = 1 - \sqrt{x^3 + 2x^2 + 2x + 4}$
	\end{itemize}
	
	WARNING: Watch out for dividing by 0! If you create a possible divide by 0 with $y$, then do two separate cases.
\end{document}