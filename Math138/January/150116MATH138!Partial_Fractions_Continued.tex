\documentclass{letter}
\usepackage[margin=0.75in]{geometry}
\usepackage{amsmath}
\usepackage{amssymb}
\usepackage{enumerate}
\usepackage{changepage}
\usepackage{tikz}
\usepackage{pgfplots}
\pgfplotsset{compat=1.8}

\pgfplotsset{vasymptote/.style={
		before end axis/.append code={
			\draw[densely dashed] ({rel axis cs:0,0} -| {axis cs:#1,0})
			-- ({rel axis cs:0,1} -| {axis cs:#1,0});
		}
	}}
	
\newcommand{\m}{\begin{bmatrix}}
\newcommand{\mm}{\end{bmatrix}}
\newcommand{\0}[1]{\begin{bmatrix}#1\end{bmatrix}}
\newcommand{\h}[1]{\underline{\textbf{#1}}}	

\begin{document}
	\begin{center}
		\LARGE Math138 - January 13'th, 2016\\
		\large Partial Fractions Continued
	\end{center}
	\vspace{0.25 in}
	
	\h{Examples:}
	
	\begin{enumerate}[1)]
		\item \begin{flalign*}
			&\int \frac{x+3}{x^4 + 9x^2}\; dx&\\
			\frac{x+3}{x^2(x^2+9)} &= \frac{A}{x} + \frac{B}{x^2} + \frac{Cx+ D}{x^2+9}\\
			\implies x+3 &= Ax^3 + 9Ax + Bx^2 + 9B + Cx^3 + Dx^2\\
			&= (A+C)x^3 + (B+D)x^2 + 9AX + 9B\\\\
			&\text{By Substitution...}\\
			0 &= A + C\\
			0 &= B + D\\
			1 = 9A \implies A &= \frac19\\
			3 = 9B \implies B &= \frac13\\\\
			\int \frac{x+3}{x^4+9x^2}\;dx &= \frac{1}{9x} + \frac{1}{3x^2} + \frac{\frac{-1}{9}x - \frac{1}{3}}{x^2 + 9}\;dx\\
			&= \frac19 \ln \mid x \mid - \frac13 x - \frac19 \int \frac{x}{x^2 - 9} - \frac13 \int \frac{1}{x^2 - 9}\;\;\;\;\text{(Use U-sub)}\\
			&= \frac19 \ln \mid x \mid - \frac{1}{3x} - \frac{1}{9}\left( \frac13 \ln \mid x^2 + 9 \mid \right) - \frac13 \left( \frac13 \arctan(\frac{x}{3})\right) + c
		\end{flalign*}
		
		\item \begin{flalign*}
			&\int \frac{x^3 - 2x}{x^2 + 3x + 2}\;dx&\\
			&\text{Notice deg(num) \textgreater\; deg(denom). We must long divide. After long dividing we get:}\\
			\frac{x^3 - 2x}{x^2 + 3x + 2} &= x - 3 + \frac{5x + 6}{x^2 + 3x + 2}\\
			\frac{5x-6}{(x+2)(x+1)} &= \frac{A}{x+2} + \frac{B}{x+1}\\\\
			&\text{We follow the same steps as before to get:}\\
			A &= 4\\
			B &= 1\\\\
			\int \frac{x^3 - 2x}{x^2 + 3x + 2} &= \int (x-3) + \frac{4}{x+2} + \frac{1}{x+1}\; dx\\
			&= \frac{x^2}{2} - 3x + 4\ln \mid x+2\mid + \ln \mid x+1\mid + c
		\end{flalign*}
	\end{enumerate}
	\clearpage
	\h{Strategy for Integration}
	
	\begin{enumerate}[\#1)]
		\item Try an algebraic manipulation (ex. Expanding, long division, factoring, identities)
		\begin{flalign*}
			a)\;&\int \frac{\tan^2 \theta}{\sec^2 \theta}\; d\theta&\\
			&= \int \frac{\sin^2 \theta}{\cos^2 \theta} \cdot \cos^2 \theta \; d\theta\\
			&= \int \sin^2 \theta\\
			&= \int \frac12\left(1 - \cos(2\theta)\right)\\
			&\text{At this point, it's a simple integral.}
		\end{flalign*}
		\begin{flalign*}
			b)\;&\int \frac{x+1}{x^2 + 4x + 3}\; dx&\\
			&= \int \frac{x+1}{(x+1)(x+3)}\\
			&= \int \frac{1}{x+3}\\
			&= \ln \mid x+3 \mid + c
		\end{flalign*}
		
		\item Look for a substitution!
		\begin{itemize}
			\item Let a troublesome term = $u$
			\item Let something inside an ugly power = $u$
			\item Let a function inside a function = $u$
		\end{itemize}
		
		\begin{flalign*}
			a)\;&\int \frac{\ln x}{x} \;\;\; u = \ln x&\\
			&= \int u \; du\\
			&= \frac{u^2}{2}\\
			&= \frac{(\ln x)^2}{2}
		\end{flalign*}
		\begin{flalign*}
			b)\;&\int e^{\sqrt{x}}\;\;\; u=\sqrt{x} &\\
			&= 2\int e^u\;du\\
			&\text{Now use IBP}
		\end{flalign*}
		\begin{flalign*}
			c)\;&\int \frac{x^2}{x^3+7}\; dx\;\;\; u = x^3 + 7&\\
			&= \int \frac{1}{3u^{9/17}}\; du\\
			&= \frac13 \int u^{-9^17}\; du\\
			&\text{Now use power rule.}
		\end{flalign*}
		\clearpage
		\item Look at major attributes of the integrand
		\begin{itemize}
			\item Powers of sin/cos or sec/tan or csc/cot?
			\begin{itemize}
				\item Use appropriate u-sub
			\end{itemize}
			\item Rational Functions?
			\begin{itemize}
				\item Long division
				\item Partial Fractions
			\end{itemize}
			\item Radicals?
			\begin{itemize}
				\item Completing the square and trig sub.
			\end{itemize}
			\item Products of unrelated functions?
			\begin{itemize}
				\item IBP (ILATE)
			\end{itemize}
		\end{itemize}
	\end{enumerate}
	
	\h{Can we Integrate Any Elementary Function?}
	
	Nope.
	
	For example, you we cannot find the anti-derivative of:
	
	$\displaystyle \frac{e^x}{x}, e^(x^2), \frac{\sin x}{x}, \frac{\cos x}{x}, \sin{x^2}, \frac{1}{\ln x}, \cos(x^2)$
\end{document}