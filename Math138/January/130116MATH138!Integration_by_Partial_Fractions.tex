\documentclass{letter}
\usepackage[margin=0.75in]{geometry}
\usepackage{amsmath}
\usepackage{amssymb}
\usepackage{enumerate}
\usepackage{changepage}
\usepackage{tikz}
\usepackage{pgfplots}
\pgfplotsset{compat=1.8}

\pgfplotsset{vasymptote/.style={
		before end axis/.append code={
			\draw[densely dashed] ({rel axis cs:0,0} -| {axis cs:#1,0})
			-- ({rel axis cs:0,1} -| {axis cs:#1,0});
		}
	}}
	
\newcommand{\m}{\begin{bmatrix}}
\newcommand{\mm}{\end{bmatrix}}
\newcommand{\0}[1]{\begin{bmatrix}#1\end{bmatrix}}
\newcommand{\h}[1]{\underline{\textbf{#1}}}	

\begin{document}
	\begin{center}
		\LARGE Math138 - January 13'th, 2016\\
		\large Integration by Partial Fractions
	\end{center}
	\vspace{0.25 in}
	
	\h{Idea:}
	
	When evaluating $\int \frac{p(x)}{d(x)}\; dx$ where $p$ and $q$ are polynomials, try decomposing the integrand into many easier pieces.
	
	In out rules, we always assume the degree of our numerator is less than the degree of the denominator. If it isn't, do long division first.
	
	\begin{tabular}{c|c}
		\textbf{If the denominator has:}&\textbf{Then write:}\\
		\hline\\
		1) Distinct linear factors & Choose one constant per factor\\
		$(a_1x+b_1)(a_2x+b_2)\dots(a_nx+b_n)$&$\frac{A_1}{a_1x+b_1}+\dots+\frac{A_n}{a_nx+b_n}$\\\\
		\hline\\
		2)Repeated linear factor & One constant per term\\
		$(ax+b)^n$&$\frac{A_n}{ax+b} + \frac{A_2}{(ax+b)^2} + \dots + \frac{A_n}{(ax+b)^n}$\\\\
		\hline\\
		3) A product of distinct irreducible quadratic factors. & A linear term per factor\\
		$(a_1x^2 + b_1x+c_1) + \dots + (a_nx^2 + b_nx + c_n)$&$\frac{A_1x + B_1}{a_1x^2+b_1x+c_1} + \dots + \frac{A_nx+B_n}{a_nx^2+b_nx+c_n}$\\\\
		\hline\\
		4) A repeated irreducible quadratic&A linear term per power\\
		$(x^2 + bx + c)^n$&$\frac{A_1x + B_1}{(ax^2+bx+c)} + \dots + \frac{A_n x + B_n}{(ax^2+bx+c)^n}$\\\\
		\hline
	\end{tabular}
	
	\h{Decomposition Practice}
	
	\begin{enumerate}[1)]
		\item $\frac{1}{(x+1)(x+2)} = \frac{A}{x+1} + \frac{B}{x+2}$
		\item $\frac{1}{x^2(x+1)} = \frac{A}{x} + \frac{B}{x^2} + \frac{C}{x+1}$
		\item $\frac{1}{x^2(x+1)^2(2x-7)} = \frac{A}{x} + \frac{B}{x^2} + \frac{C}{x+1} + \frac{D}{(x+1)^2} + \frac{E}{2x-7}$
	\end{enumerate}
	
	To find $A, B, C, \dots$ we either sub good values of $x$ in or cross multiply and compare coefficients.
	\clearpage
	\h{Examples:}
	
	\begin{enumerate}
		\item \begin{flalign*}
			&\int \frac{x}{x^2 - 4x + 5}\; dx&\\
			&\frac{x}{(x-5)(x+1)} = \frac{A}{x+1} + \frac{B}{x-5}\\
			&\implies x = (x+1)(x-5)\left[ \frac{A}{x+1} + \frac{B}{x-5}\right]\\
			&= (x-5)A + (x+1)B
		\end{flalign*}
		
		\h{Finding A and B}
		
		Method \#1: Compare Coefficients:
		\begin{flalign*}
			x &= Ax - 5A + Bx + B&\\
			&= (A+B)x - 5A + B\\\\
			A+B &= 1\\
			0 &= -5A + B\\
			B &= 5A\\
			A +5A &= 1\\
			6A &= 1\\
			A &= \frac16\\
			\frac16 + &= 1\\
			B &= \frac56
		\end{flalign*}
		
		Method \# 2: Put in smart $x$ values:\\\\
		The goal here is to insert $x$ such that A or B cancels out.\\\\
		$x=(x-5)A + (x+1)B$\\\\
		Sub in $x = 5$:\\
		$5 = 0 + 6B$\\
		$B = \frac56$\\\\
		Sub in $x = -1$\\
		$-1 = -6A + 0$\\
		$A = \frac16$\\\\
		So,
		\begin{flalign*}
			\int \frac{x}{(x+1)(x-5)} dx &= \int \frac{1/6}{x+1} + \frac{5/6}{x-5} dx&\\
			&= \frac16 \int{1}{x+1} + \frac56 \int{1}{x-5}\\
			&= \frac16 \ln\mid x+1\mid + \frac56 \ln \mid x-5 \mid + c
		\end{flalign*}
	\end{enumerate}
\end{document}