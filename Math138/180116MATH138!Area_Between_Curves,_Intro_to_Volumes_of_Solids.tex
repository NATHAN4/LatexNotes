\documentclass{letter}
\usepackage[margin=0.75in]{geometry}
\usepackage{amsmath}
\usepackage{amssymb}
\usepackage{enumerate}
\usepackage{changepage}
\usepackage{tikz}
\usepackage{pgfplots}
\pgfplotsset{compat=1.8}

\pgfplotsset{vasymptote/.style={
		before end axis/.append code={
			\draw[densely dashed] ({rel axis cs:0,0} -| {axis cs:#1,0})
			-- ({rel axis cs:0,1} -| {axis cs:#1,0});
		}
	}}
	
\newcommand{\m}{\begin{bmatrix}}
\newcommand{\mm}{\end{bmatrix}}
\newcommand{\0}[1]{\begin{bmatrix}#1\end{bmatrix}}
\newcommand{\h}[1]{\underline{\textbf{#1}}}	

\usepackage{listings}
\usepackage{color}
\definecolor{lightgray}{rgb}{.9,.9,.9}
\definecolor{darkgray}{rgb}{.4,.4,.4}
\definecolor{purple}{rgb}{0.65, 0.12, 0.82}

\begin{document}
	\begin{center}
		\LARGE Math138 - January 18'th, 2016\\
		\large Area Between Curves, Intro to Volume of Solids
	\end{center}
	\vspace{0.25 in}
	
	\h{Area's Between Curves}
	
	Suppose we want to find the area between two curves$f(x)$ and $g(x)$.
	
	The area is = area under $f(x)$ - area under $g(x)$, so long as $f(x)$ is on top.
	
	The formula for the area bounded by $f$ and $g$ ($f \geq g$) between $x=a$ and $x=b$ is:
	
	$\displaystyle \int_a^b f(x) - g(x)\;dx$
	
	\begin{itemize}
		\item[Ex. ] Find the area between $f(x) = x^2$ and $g(x) = x$ from $x=1$ to $x=3$.\\\\
		By a sketch, we see $x^2$ is above $x$ between $x=1$ and $x=3$.
		\begin{flalign*}
			A &= \int_1^3 x^2 - x\;dx&\\
			&= \left[ \frac{x^3}{3} - \frac{x^2}{2} \right]_1^3\\
			&= \vdots \;\;\;\;\text{(Note: I'll start omitting more steps because they're simple and I'm lazy)}
			&= \frac{14}{3}
		\end{flalign*}
	\end{itemize}
	
	\h{Negative Area}
	
	If you get negative area between curves, you screwed up! Check your integral, and check you have the correct function on top.
	
	Sometimes, we'll want to find the area between curves on an interval where they intersect and the function on top changes. For this case, we need a more accurate formula for area between curves.
	
	\h{The Official Actual For-Real Formula For Area Between Curves}
	
	\begin{flalign*}
		A &= \int_a^b \mid f(x) - g(x) \mid dx &
	\end{flalign*}
	
	It's hard to work with absolute values though, so we'll split the integral up where the functions intersect instead.
	
	\begin{itemize}
		\item[Ex. ] Find the area enclosed by:
		
		$\displaystyle f(x) = 1-x^2\;\;\;g(x) = x^2$\\\\
		By doing a sketch, we see $f(x)$ is on top. We need to find our bounds of integration though, this question didn't give us any. To do that, we set the functions equal to find the intersections.\\\\
		$x = \pm \frac{1}{\sqrt2}$\\
		\begin{flalign*}
			A &= \int_{\frac{-1}{\sqrt2}}^{\frac{1}{\sqrt2}} (1-x^2) - x^2\;dx&\\
			&= \vdots\\
			&=\sqrt{2}
		\end{flalign*}
	\end{itemize}
	\clearpage
	\h{Functions of $y$}
	
	When taking an integral of a function of $y$, we do $\displaystyle \int_a^b $right - left
	
	\begin{itemize}
		\item[Ex. ] Find the area bounded by $x=y^2 + 1, x=y$ from $y=0$ to $y=2$.\\\\
		By sketch, we see $x = y^2 +1$ is the outer/right curve.
		\begin{flalign*}
			A &= \int_0^2 y^2 + 1 - y\;dy\\
			&= \vdots\\
			&= \frac83
		\end{flalign*}
	\end{itemize}
	
	\h{Volumes of Revolution}
	
	Goal: Learn how to find the volume of certain symmetric objects using fancy integration techniques.
	
	For more general objects, we need even fancier integration techniques. For these, see multi-variable calculus in MATH237.
	
	The general idea is similar to how we found the area under a curve before. We will split up our object into infinitely many infinitely small slices. At $x$, the slice has area $A(x)$.
	
	Item volume $\displaystyle = \int_a^b A(x)\;dx$
	
	\begin{itemize}
		\item[Ex. ] Find the volume of a sphere with radius $r$.\\\\
		Note that using the formula $x^2 + y^2 = r^2$, we can find out the x, y, or radius of a circle using the other two pieces of information. If we manipulate this function into $y = \sqrt{r^2 - x^2}$, we have a function for $y$ given the radius and an $x$ value.\\\\
		For a sphere centered on the origin, the $y$ value at a given $x$ represents the radius (Remember half of the sphere is BELOW the x-axis). Now we can create a formula that defines the area of a 'slice'.\\
		\begin{flalign*}
			A(x) &= \text{Area of circle @ x}&\\
			&= \pi(\sqrt{r^2 - x^2})^2\\
			&= \pi(r^2 - x^2)
		\end{flalign*}
		
		Therefore,
		
		\begin{flalign*}
			\text{Volume} &= \int_{-r}^{r} \pi(r^2 - x^2)\; dx&\\
			&= \vdots\\
			&= \frac43 \pi r^3
		\end{flalign*}
	\end{itemize}
	
	There are two methods we will use to calculate the value of a solid of revolution:
	
	\begin{enumerate}[\#1)]
		\item Washers/Discs (A.K.A cross sections)
		\item Cylindrical Shells
	\end{enumerate}
	
	Each has it's uses, sometimes only one method of the two will actually work.
\end{document}