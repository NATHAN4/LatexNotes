\documentclass{letter}
\usepackage[margin=0.75in]{geometry}
\usepackage{amsmath}
\usepackage{amssymb}
\usepackage{enumerate}

\begin{document}
	\begin{center}
		\large MATH135 Fall 2015 Assignment 9, Due at 8:25am on Wednesday November 15, 2015\\
	\end{center}
	\vspace{0.25 in}
	\textbf{I.D. Number:} 20612050 \textbf{First Name:} Nathaniel \textbf{Family Name:} Woodthorpe
	
	\textbf{List any references that you used beyond the course text and lectures (e.g. discussions, texts, or online resources). If you did not use any aids, state this is in the space provided.}
	
	I did not use any aids.
	
	
	\clearpage
	
	\large 1. Find $m \in \mathbb{R}$ such that the equation $2z^2 - (3-3i)z - (m-9i)=0$ has a real root.
	
	\hrulefill\\
	
	Note: In order for us to have a real root, we need to find a real value of $z$ that cancels out all $i$ terms.
	
	Suppose $z=-3$
	
	\begin{flalign*}
		2(-3)^2 - (3-3i)(-3) - (m-9i) &= 0&\\
		18 + 9 - 9i -m + 9i &= 0\\
		27 - m &= 0
	\end{flalign*}
	
	As we can see here, should $m = 27$, we would have a real root at $z = -3$.
	
	$\therefore m = 27$ gives the above equation a real root.
	
	\clearpage
	
	2. Let $z \in \mathbb{C}$. Draw the region of the complex plane for which $\dfrac{iz - 1}{z-i}$ is real.
	
	\hrulefill\\
	
	Let $z = x + yi$
	
	\begin{flalign*}
		\dfrac{i(x + yi) - 1}{x + yi - i} &= \dfrac{xi - y - 1}{x + yi - i}&\\
		&= \bigg( \dfrac{xi - y - 1}{x + (y-1)i} \bigg) \bigg( \dfrac{x - (y-1)i}{x- (y-1)i} \bigg)\\
		&= \dfrac{(xi - y - 1) (x-(y-1)i)}{x^2 + (y-1)^2}\\
		&= \dfrac{x^2i + x(y-1) - xy + y(y-1)i -x + (y-1)i}{x^2 + y^2 - 2y + 1}\\
		&= \dfrac{x^2i + y^2i - i -2x}{x^2 + y^2 -2y + 1}\\
		&= \dfrac{i(x^2 + y^2 - 1) - 2x}{x^2 + y^2 - 2y + 1}\\
	\end{flalign*}
	
	For this number to be real, we must have no $i$ component. In other words, $x^2 + y^2 - 1 = 0$, or $x^2 + y^2 = 1$.
	
	This means $z$ must lie on the circle of radius 1 centered at the pole, or take the form\\ $z = 1(cos k \theta + i \sin k \theta)$ for $k \in \mathbb{R}$
	
	\clearpage
	
	3. Suppose $a, b,$ and $z$ are complex numbers. Prove that if $z$ satisfies the equation
	
	\[ \vert z - a \vert = \vert z - b \vert \]
	
	Then there exists a $w \in \mathbb{C}$ and a $x \in \mathbb{R}$ so that
	
	\[ \overline{w}z + w\overline{z} = x \]
	
	\hrulefill\\
	
	We need to find a $w$ that makes $\overline{w}z + w\overline{z}$ real. Suppose $w = z$
	
	We get:\\ $\overline{z}z + z\overline{z}$\\
	$= \vert z \vert^2 + \vert z \vert^2$\\
	$\vert z \vert$ is a real number, thus $\vert z \vert ^2 + \vert z \vert ^2$ is a real number.
	
	$\therefore$ if $w = z$ and $x = 2\vert z \vert^2$, the above equation is satisfied.
	
	\clearpage
	
	4. Suppose $z, w \in \mathbb{C}$. Prove that $\vert z - w \vert^2 = \vert z \vert^2 + \vert w \vert^2 - 2$Re$(z\overline{w})$
	
	\hrulefill\\
	
	LS: $\vert z - w \vert^2$\\
	$= (\overline{z - w})(z-w)$\\
	$= (\overline{z} - \overline{w})(z-w)$\\
	$= \overline{z}z - - \overline{z}w - \overline{w}z + \overline{w}w$\\
	$= \vert z \vert^2 + \vert w \vert^2 - \overline{z}w - \overline{w}z$\\
	$= \vert z \vert^2 + \vert w \vert^2 - (z\overline{w} + \overline{z}w)$\\
	$= \vert z \vert^2 + \vert w \vert^2 - (z\overline{w} + \overline{z\overline{w}})$\\
	$=\vert z \vert^2 + \vert w \vert^2 - 2$Re$(z\overline{w})$\\
	$= $RS\\
	QED
	
	\clearpage
	
	5. Determine all $\theta \in \mathbb{R}$ satisfying
	\[ \prod_{k=1}^{100} (\cos(k \theta) + i \sin(k \theta)) = 1 \]
	Justify your answer.
	
	\hrulefill\\
	
	When we multiply complex numbers in polar form with modulus 1, we sum their angle arguments (By DMT). Thus,
	
	\begin{flalign*}
		\prod_{k=1}^{100} (\cos(k \theta) + i \sin(k \theta)) &= 1 &\\
		\cos(\sum_{k=1}^{100} k\theta) + i \sin(\sum_{k=1}^{100} k\theta) &= 1\\
		\cos(\theta\sum_{k=1}^{100} k) + i \sin(\theta\sum_{k=1}^{100} k) &= 1\\
		\cos(\dfrac{(100)(101)\theta}{2}) + i \sin(\dfrac{(100)(101)\theta}{2}) &= 1\\
		\cos(50(101)\theta) + i \sin(50(101)\theta) &= 1\\
		\cos(5050\theta) + i \sin(5050\theta) &= \cos 0 + i \sin 0\\\\
		5050\theta &= 0 + 2m\pi, m \in \mathbb{Z}\\
		\therefore \theta &= \dfrac{m\pi}{2525}
	\end{flalign*}
	
	\clearpage
	
	6. Compute $(2-2\sqrt3 i)^{20}$. Write your answer in standard form.
	
	
	\hrulefill\\
	
	Lets start by expressing $2 - 2\sqrt3 i$ in polar form.
	
	$x= 2, y = -2\sqrt3$\\
	\begin{minipage}[t]{0.5\textwidth}
		\begin{flalign*}
			r &= \sqrt{x^2 + y^2}&\\
			&= \sqrt{4 + 12}\\
			&= 4
		\end{flalign*}
	\end{minipage}
	\begin{minipage}[t]{0.5\textwidth}
		\begin{flalign*}
			\theta &= \arctan \frac{y}{x}]&\\
			&= \arctan -\sqrt3\\
			&= \frac{-\pi}{3}
		\end{flalign*}
	\end{minipage}
	\begin{flalign*}
	(2-2\sqrt3 i)^{20} &= 4(\cos \frac{-\pi}{3} + i \sin \frac{- \pi}{3})^{20}&\\
	&= 4^{20}(\cos \frac{-20\pi}{3} + i \sin \frac{-20 \pi}{3})\\
	&= 4^{20} (\cos \frac{4\pi}{3} + i \sin\frac{4 \pi}{3})\\
	&= 4^{20}(\frac{-1}{2} + i \frac{-\sqrt3}{2})\\
	&= -2(4^{19}) + -2i\sqrt3(4^{19})
	\end{flalign*}
	
	\clearpage
	
	7. Suppose $n \in \mathbb{N}$ and $z \in \mathbb{C}$ with $\vert z \vert = 1$ and $z^{2n} \neq -1$. Prove that $\frac{z^n}{1 + z^{2n}} \in \mathbb{R}$
	
	\hrulefill\\
	
	Let $z = (\cos \theta+ i \sin \theta)$
	
	
	\begin{flalign*}
		\dfrac{(\cos \theta + i \sin \theta)^n}{1 + (\cos \theta + i \sin \theta)^{2n}} &= \dfrac{\cos (n \theta) + i \sin(n \theta)}{1 + \cos(2n\theta) + i \sin(2n \theta)}&\\
		&= \dfrac{\cos(n\theta) + i \sin(n\theta)}{1 + 2\cos^2(n\theta) - 1 + 2i \sin(n\theta)\cos(n\theta)}\\
		&= \dfrac{\cos(n \theta) + i \sin(n\theta)}{2\cos(n \theta)(\cos(n\theta) + i \sin(n\theta))}\\
		&= \dfrac{1}{2 \cos(n\theta)}
	\end{flalign*}
	
	There is no imaginary part, therefore this number is $\in \mathbb{R}$
	
	\clearpage
	
	A blank page for Crowdmark's pleasure.
	
	\clearpage
	
	A blank page for Crowdmark's pleasure.
\end{document}