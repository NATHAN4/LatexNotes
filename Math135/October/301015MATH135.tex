\documentclass{letter}
\usepackage[margin=0.75in]{geometry}
\usepackage{amsmath}
\usepackage{amssymb}
\usepackage{enumerate}

\begin{document}
	\begin{center}
		\LARGE Math135 - October 29'th, 2015
	\end{center}
	\vspace{0.25 in}
	\underline{\textbf{Everything We Know About Congruences}}
	\vspace{0.25 in}
	
	$a \equiv r \;(\bmod\; n)$
	\begin{itemize}
		\item $\iff m \mid (a-b)$
		\item $a \bmod m = b \bmod m$
		\item Transitive:\\
		$\;[a \equiv b \;(\bmod\; m)\;]
		\land 
		\;[b \equiv c \;(\bmod\; m)\;]
		\implies
		a \equiv c \;(\bmod\; m) $
		\item Symetric: \\
		$a \equiv b\;(\bmod\; m) 
		\iff
		b \equiv a\;(\bmod\; m)$
		\item Reflective:\\
		$a \equiv a\;(\bmod\; m)$
		\item You can multiply, add, or subtract $a$ to the left side and $b$ to the right side so long as 
		$a \equiv b\;(\bmod\; m)$
		\item You can divide both sides by $n$ if $\gcd(n, m) = 1$ (Co-prime)
	\end{itemize}
	
	\vspace{0.25 in}
	
	\underline{\textbf{Congruent Iff Same Remainder (CISR)}}
	\vspace{0.25 in}

	Let $a, b, c \in \mathbb{N}$ where $m>0$\\
	$a \equiv b\;(\bmod\; m) \iff a$ and $b$ have the same remainder when divided by $m$.
	
	\vspace{0.25 in}
		
	\underline{\textbf{Linear Congruence}}
	\vspace{0.25 in}
	
	Let $a, c, m \in \mathbb{Z}$ where $m>0$
	A relation of the form 

	\begin{center}
	$ax \equiv c\;(\bmod\; m)$
	\end{center}
	
	
	is called a Linear Congruence in the variable $x$. A solution is an integer $ax_0 \equiv c\;(\bmod\; m)$
	
	\vspace{0.25 in}
			
	\underline{\textbf{Examples:}}
	\vspace{0.25 in}
	
	\begin{minipage}[t]{0.5\textwidth}
		\begin{enumerate}[i)]
	  		\item 
	  		\begin{math}
	  			4x \equiv 5\;(\bmod\; 8)\\
	  			4x - 5 = 8k\\
	  			4x - 8k = 5\\
	  			\gcd(4, -8) = 4\\
	  			4 \nmid -8\\
	  			\therefore \text{No solution}
	  		\end{math}
		\end{enumerate}
	\end{minipage}
	\begin{minipage}[t]{0.5\textwidth}
		\begin{enumerate}[i)]
			\setcounter{enumi}{1}
			\item
	        \begin{math}
			  	5x \equiv 3\;(\bmod\; 7)\\
			  	5x - 3 = 7k\\
			  	5x - 7k = 3\\
			  	5x + 7y = 3  \text{ (Let }y = -k)\\
			  	x=2, y=1\\
			  	x = 2 + 7n \text{ (By LDET2)}\\
			  	\therefore x \equiv 2\;(\bmod\; 7)  
			\end{math}
		\end{enumerate}
	\end{minipage}\\
	\clearpage
	\begin{enumerate}[i)]
		\setcounter{enumi}{2}
		\item
		\begin{math}
			2x \equiv 4 \;(\bmod \;6)\\
			2x - 4 = 6k\\
			2x - 6k = 4\\
			2x + 6y = 4\\
			\gcd(2, 6) = 2\\
			2 \vert 4 \text{ Thus there is a solution}\\
			\text{By inspection, } x=-1, y=1\\
			x = -1 + 3n \text{ By LDET2}\\
			x \equiv -1 \;(\bmod\; 3)\\
			x \equiv 3 \;(\bmod\; 3)
		\end{math}
	\end{enumerate}

	\vspace{0.25 in}
				
	\underline{\textbf{Generalised Linear Congruence Rules:}}
	\vspace{0.25 in}
	
	\begin{enumerate}[i)]
		\item A solution does not always exist.
		\item $\gcd(a, m) \vert c$ means a solution exists.
		\item To find a particular solution, convert into a linear diophantine equation using:\\
		$ax \equiv c\;(\bmod\; m) \implies (ax- c) = mk, k \in \mathbb{N}$
	\end{enumerate}
	
	\vspace{0.25 in}
					
	\underline{\textbf{Generalised Linear Congruence Rules:}}
	\vspace{0.25 in}
		
	Example) Solve $x^2 \equiv 6\;(\bmod\; 10)$\\	
	\begin{enumerate}[-]
		\item There is no efficient way to solve polynomial congruences.
	\end{enumerate}
	
	\begin{minipage}[t]{0.2\textwidth}
		\begin{math}
		 	x\;(\bmod\; 10)\\
		  	x^2\;(\bmod\; 10)
		\end{math}
	\end{minipage}
	\begin{minipage}[t]{0.8\textwidth}
		\begin{text}
			0 1 2 3 4 5 6 7 8 9\\
			0 1 4 9 6 5 6 9 4 1
		\end{text}
	\end{minipage}
	
	\begin{enumerate}[-]
		\item As we can see, $x \equiv 4, 6 \;(\bmod\;  10)$
	\end{enumerate}	
\end{document}