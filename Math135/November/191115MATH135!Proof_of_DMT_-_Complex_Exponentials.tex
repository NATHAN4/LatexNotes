\documentclass{letter}
\usepackage[margin=0.75in]{geometry}
\usepackage{amsmath}
\usepackage{amssymb}
\usepackage{enumerate}
\usepackage{changepage}

\begin{document}
	\begin{center}
		\LARGE Math135 - November 19, 2015\\
		\large Proof of DMT - Complex Exponentials
	\end{center}
	\vspace{0.25 in}
	\underline{\textbf{Imaginary Inverse Property:}}
	
	\[ z^{-n} = (z^{-1})^n = (z^n)^{-1} = \dfrac{1}{z^n} \]
	
	\underline{\textbf{Proof of DMT}}
	
	Reminder: DMT States that $(\cos \theta + i \sin \theta)^n =
	 (\cos (n \theta) + i \sin (n \theta))$
	\begin{itemize}
		\item[Case 1: ]\textbf{n = 0}\\\\
		\begin{minipage}[t]{0.5\textwidth}
			\begin{itemize}
				\item[LS: ] $(\cos \theta + i \sin \theta)^0$\\
				$=1$
			\end{itemize}
		\end{minipage}
		\begin{minipage}[t]{0.5\textwidth}
			\begin{itemize}
				\item[RS: ] $\cos 0 + i \sin 0$\\
				$= 1 + 0i$\\
				$= 1$\\
				$= LS$
			\end{itemize}
		\end{minipage}
		\\
		\item[Case 2: ] $\mathbf{n > 0}$\\\\
		Proof by induction:\\\\
		Base Case: $n = 1$\\
		\begin{itemize}
			\item[\;\;\;]
			\begin{minipage}[t]{0.5\textwidth}
				\begin{itemize}
					\item[LS: ] $(\cos \theta + i \sin \theta)^1$\\
					$= \cos \theta + i \sin \theta$
				\end{itemize}
			\end{minipage}
			\begin{minipage}[t]{0.5\textwidth}
				\begin{itemize}
					\item[RS: ]$\cos((1)\theta) + i \sin((1)\theta)$\\
					$= \cos \theta + i \sin \theta$\\
					$= LS$
				\end{itemize}
			\end{minipage}
		\end{itemize}
		Inductive Hypothesis:
		\begin{itemize}
			\item[\;\;\;] Assume $(\cos \theta + i \sin \theta)^k = \cos(k \theta) + i \sin(k \theta)$ for some $k > 0$
		\end{itemize}
		Inductive Conclusion:
		\begin{itemize}
			\item[\;\;\;] Need to show: $(\cos \theta + i \sin \theta)^{k+1} = \cos((k+1)\theta) + i \sin((k+1)\theta)$\\
			\begin{itemize}
				\item[LS: ]
				\begin{flalign*}
					&(\cos \theta + i \sin \theta)^{k+1}&\\
					&= (\cos \theta + i \sin \theta)^k (\cos \theta + i \sin \theta)\\
					&= (\cos (k \theta) + i \sin (k \theta))(\cos \theta + i \sin \theta) \;\;\;\text{(By Inductive Hypothesis)}\\
					&=\cos (k\theta + k) + i \sin(k \theta + k) \;\;\; \text{(By PMCT)}\\
					&= \cos((k+1)\theta) + i \sin ((k+1)\theta)\\
					&= RS
				\end{flalign*}
			\end{itemize}
		\end{itemize}
		
		$\therefore$ by POMI, $(\cos \theta + i \sin \theta)^n = \cos(n \theta) + i \sin (n \theta)$ for all $n > 0$.
		\clearpage
		\item[Case 3: ] $\mathbf{n < 0}$\\
		
		Let $n = -m, m > 0$
		
		\begin{flalign*}
			(\cos \theta + i \sin \theta)^n &= (\cos \theta + i \sin \theta)^{-m}&\\
			&= \dfrac{1}{(\cos \theta + i \sin \theta)^m}\\
			&= \dfrac{1}{(\cos(m \theta) + i \sin(m \theta))}\;\;\; \text{(By Case 2)}\\
			&= \dfrac{\cos(m \theta) - i \sin(m \theta)}{\cos^2(m \theta) + \sin^2(m \theta)}\\
			&= \cos(-m \theta) + i \sin(-m \theta)\;\;\; \text{(By symmetry of sin/cos)}\\
			&=\cos(n \theta) + i \sin(n \theta)
		\end{flalign*}
	\end{itemize}
	QED. 
	
	\underline{\textbf{A Proof with Imaginary Numbers}}
	
	\begin{itemize}
		\item[Ex. ] Prove that if $w$ is complex, $\vert w \vert = 1$ and $\theta$ is an argument of $w$, then $\frac{-i}{2}(w^n - w^{-n}) = \sin (n \theta)\; \forall\; n \in \mathbb{Z}$\\\\
		\begin{flalign*}
			\text{LS: } &= \frac{-i}{2}\left( w^n - w^{-n} \right)&\\
			&= \frac{-i}{2} \left( (\cos \theta + i \sin \theta)^n - (\cos \theta + i \sin \theta)^{-n} \right)\\
			&= \frac{-i}{2} \left (\cos (n \theta) + i \sin (n \theta) - (\cos(-n \theta) + i \sin(-n \theta))\right)\\
			&= \frac{i}{2} \left( \cos (n \theta) + i \sin (n \theta) - \cos(n \theta) + \sin(n \theta)\right)\\
			&= \frac{-i}{2} \left( 2i \sin(2 \theta) \right)\\
			&= \sin (n \theta)\\
			&= \text{RS} \;\;\;\; \square
		\end{flalign*}
	\end{itemize}
	
	\underline{\textbf{Complex Exponential}}
	
	We will define complex exponential functions as:
	
	\[ e^{i \theta} = \cos \theta + i \sin \theta \]
	
	We can write the polar form of any complex number as $z = 4 e^{i \theta}$, where $r = \vert z \vert$, and $\theta$ is an argument of $z$.
	
	\begin{itemize}
		\item[Ex. ] Write $(2 e^{i(\frac{11 \pi}{6})})^6$ in standard form. 
		\begin{flalign*}
			(2 e^{i(\frac{11 \pi}{6})})^6 &= (2(\cos \frac{11 \pi}{6} + i \sin \frac{11 \pi}{6}))^6&\\
			&= 2^6 (\cos 11\pi + i \sin 11\pi)\\
			&= 64(\cos \pi + i \sin \pi)\\
			&= 64(-1 + 0i)\\
			&= -64
		\end{flalign*}
	\end{itemize}
	\clearpage
	\underline{\textbf{Awesome Properties of Complex Exponential Form}}
	
	Notice how $(2 e^{i \frac{11 \pi}{6}})^6 = 2^6 e^{i 11 \pi}$.
	
	Exponential form is useful because most of the exponent laws from the reals still apply.
	
	\begin{itemize}
		\item[Ex. ] $(e^{i \theta})^n = e^{in\theta}$\\
		$(e^{i\theta})(e^{i\phi}) = e^{i(\theta + \phi)}$
	\end{itemize}
	
	Derivatives also work similarly in complex exponential form.
	
	$\dfrac{d}{d\theta} (e^{i \theta}) = i(e^{i\theta})$\\
	
	\underline{\textbf{The Craziest Equation In Math}}
	
	Notice what happens when we set $r=1$, $\theta = \pi$.
	
	\[ e^{i \pi} = \cos \pi + i \sin \pi \]
	\[ e^{i \pi} = -1 \]
	\[ e^{i \pi} + 1 = 0 \]
\end{document}