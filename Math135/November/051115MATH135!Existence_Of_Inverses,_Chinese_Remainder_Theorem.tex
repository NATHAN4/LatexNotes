\documentclass{letter}
\usepackage[margin=0.75in]{geometry}
\usepackage{amsmath}
\usepackage{amssymb}
\usepackage{enumerate}

\begin{document}
	\begin{center}
		\LARGE Math135 - November 5'th, 2015\\
		\large Existence of Inverses, Chinese Remainder Theorem
	\end{center}
	\vspace{0.25 in}
	\underline{\textbf{Corollary to F$\ell$T}}
	
	For any integer $a$ and prime $p$,\\
	$a^p = a\;(\bmod\; p)$\\
	
	\underline{\textbf{Existence of Inverses in $\mathbb{Z}_p$}}
	
	Let $p$ be a prime number. If $[a]$ is any non-zero element in $\mathbb{Z}_p$, then there exists an element $[b] \in \mathbb{Z}_p$ such that $[a][b] = [1]$\\\\
	Proof:
	\begin{enumerate}[ ]
		\item Assume $[a]$ is a non-zero element in $\mathbb{Z}_p$.	\\
		$a \not\equiv 0\;(\bmod\; p)$\\
		So, $p \nmid a$\\
		By F$\ell$T, $a^{p1} \equiv 1\;(\bmod\; p)$\\
		Consider $[b] = [a^{p-2}]$ (Allowed since $p \geq 2$)\\
		$[a][b] = [a][p-2]$\\
		$[a][b] = [a^{p-1}] = [1]$
	\end{enumerate}
	This proof gives us another method to find the inverse of an element in $\mathbb{Z}_p$ if $p$ is prime.
	
	$[a^{-1}] = [a]^{p-2}$
	
	Example: What is the inverse of $7$ in $\mathbb{Z}_{11}$?
	\begin{flalign*}
		[7]^{-1} &= [7]^{11-2} &\\
		&= [7]^9\\
		&= [5^4 \cdot 7] \text{ (Because $7^2 \equiv 5\;(\bmod\;11)$)}\\
		&= [3^2 \cdot 7] \text{ (Because $5^2 \equiv 3 \;(\bmod\; 11)$)}\\
		&= [63]\\
		&= [8]
	\end{flalign*}
	
	\underline{\textbf{Examples of F$\ell$T Proofs}}
	
	Let $p$ be prime, $r, k, s \in \mathbb{Z}$:
	\begin{enumerate}[i)]
		\item If $p \nmid a$ and $r \equiv s\;(\bmod\; p-1)$, then $a^r \equiv a^s \;(\bmod\; p)$.
		
		Assume $p \nmid a$ and $r \equiv s \;(\bmod\; p)$
		
		$r - s = (p-1)k$, $k \in \mathbb{Z}$\\
		$r = (p-1)k + s$
		
		\begin{flalign*}
			a^r &\equiv a^{(p-1)k + s} \;(\bmod\; p)&\\
			&\equiv a^{(p-1)k}a^s \;(\bmod\; p)\\
			&\equiv 1^k a^s \;(\bmod\; p)\\
			&\equiv a^s \;(\bmod\; p)\\
			&\equiv a^s \;(\bmod\; p)
		\end{flalign*}
		\pagebreak
		\item If $r = pk + s$, then $a^4 \equiv a^{s+k} \;(\bmod\; p)$.
		
		\begin{flalign*}
			a^r &\equiv a^{pk + s} \;(\bmod\; p)&\\
			&\equiv (a^p)^k a^s \;(\bmod\; p)\\
			&\equiv a^k a^s \;(\bmod\; p)\\
			&\equiv a^{k + s} \;(\bmod\; p)
		\end{flalign*}
	\end{enumerate}
	
	\underline{\textbf{Chinese Remainder Theorem}}
	
	Let $a_1, a_2 \in \mathbb{Z}$. If $\gcd(m_1, m_2) = 1$, then the simultaneous linear congruences:
	
	$n \equiv a_1 \;(\bmod\; m_1)$\\
	$n \equiv a_2 \;(\bmod\; m_2)$
	
	have a unique solution modulo $(m_1)(m_2)$. Thus, if $n = n_0$ is one integer solution, then the complete solution is:
	
	$n \equiv n_0 \;(\bmod\; m_1m_2)$
	
\end{document}