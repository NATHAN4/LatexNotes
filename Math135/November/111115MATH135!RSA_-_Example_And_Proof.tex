\documentclass{letter}
\usepackage[margin=0.75in]{geometry}
\usepackage{amsmath}
\usepackage{amssymb}
\usepackage{enumerate}

\begin{document}
	\begin{center}
		\LARGE Math135 - November 11'th, 2015\\
		\large RSA - Example And Proof
	\end{center}
	\vspace{0.25 in}
	\underline{\textbf{Example:}}
	
	Suppose Alice chooses $p=11$, $q=13$, $e=23$.
	\begin{enumerate}[1)]
		\item What is Alice's public key?
		\item What is Alice's private key?
		\item If Bob wants to send message $M=25$ to Alice, what is ciphertext $C$?
	\end{enumerate}
	\begin{enumerate}[1)]
		\item Alice's public key is $(e, n)$\\
		$n=pq$\\
		$n = (11)(13)$\\
		$n = 143$\\
		Alice has already chosen $e$ such that $1 < e < (p-1)(q-1)$ and $\gcd(e, (p-1)(q-1)) = 1$.\\
		So, public key is $(23, 143)$
		\item To get private key $d$, we must solve\\
		$ed \equiv 1\;(\bmod\; 120)$\\
		$23d \equiv 1 \;(\bmod\; 120)$\\
		
		We'll use EEA to solve this.
		\begin{flalign*}
			23d &\equiv 1\;(\bmod\; 120)&\\
			23d - 1 &= 120k, k \in \mathbb{Z}\\
			120k + 23d &= 1\\
		\end{flalign*}
		\begin{tabular}{c|c|c|c}
			k&d&r&q\\
			\hline
			1&0&120&0\\
			0&1&23&0\\
			1&-5&5&5\\
			-4&21&3&4\\
			5&-26&2&1\\
			9&47&1&1\\
			23&-120&0&2
		\end{tabular}
		\;\;\;\;\;\;\;\;\;\;\;So, $d=47$.\\
		
		\item To encode the message, we need to solve the following congruence:
		\begin{flalign*}
			25^{23} &\equiv C\;(\bmod\; 143)&\\
			25^{16} 25^4 25^2 25 &\equiv C\;(\bmod\; 143) \text{ (Calculate these off to the side)}\\
			(14)(92)(53)(25) &\equiv C\;(\bmod\; 143)\\
			(1288)(1325) &\equiv C\;(\bmod\; 143)\\
			(1)(38) &\equiv C\;(\bmod\; 143)\\
			38 &\equiv C\;(\bmod\; 143)
		\end{flalign*}
		Because $C < 143$, $C = 38$.\\
		
		\textbf{Note:} If Alice wanted to decrypt message $C$, she must solve:\\
		$38^{47} \equiv 25\;(\bmod\; 143)$
	\end{enumerate}
	\pagebreak
	\underline{\textbf{RSA Theorem}}
	
	If:
	\begin{enumerate}[ 1)]
		\item $p$ and $q$ are prime numbers.
		\item $n=pq$
		\item $e$ and $d$ are positive integers such that $ed \equiv 1\;(\bmod\; (p-1)(q-1))$
		\item $0 \leq M \leq n$
		\item $M^e \equiv C\;(\bmod\; n)$
		\item $C^d \equiv R\;(\bmod\; n)$ Where $0 \leq R \leq n$
	\end{enumerate}
	Then, $R=M$.
	
	\underline{\textbf{RSA Theorem - Proof}}
	
	Assume all hypothesis of the RSA Theorem (1-6).
	\begin{flalign*}
		R &\equiv C^d\;(\bmod\; n) \text{ (By hypothesis)}&\\
		&\equiv M^{e^d}\;(\bmod\; n) \text{ (By hypothesis)}\\
		&\equiv M^{ed}\;(\bmod\; n)
	\end{flalign*}
	We know:
	\begin{flalign*}
		ed &\equiv 1\;(\bmod\; (p-1)(q-1))&\\
		&= 1 + k(p-1)(q-1), k \in \mathbb{Z}
	\end{flalign*}
	\begin{flalign*}
		\text{So, } R &\equiv M^{1+k(p-1)(q-1)}\;(\bmod\; n)&\\
		&\equiv M \cdot M^{k(p-1)q-1)}\;(\bmod\; n)\\
	\end{flalign*}
	Since $p\mid n$ and $q \mid n$, we know:\\
	
	$ \equiv  M \cdot M^{k(p-1)(q-1)}\;(\bmod\; p)$ and $R \equiv M \cdot M^{k(p-1)(q-1)}\;(\bmod\; q)$\\
	
	We will show $ \equiv M\;(\bmod\; p)$ and $R \equiv M\;(\bmod\; q)$\\
	
	\begin{minipage}[t]{0.5\textwidth}
		\textbf{Case 1:} $p \mid m$\\
		$M \equiv 0\;(\bmod\; p)$\\
		and $m \cdot M^{k(p-1)(p=1)} \equiv 0 \;(\bmod\; p)$\\
		So, $R \equiv M\;(\bmod\; p)$\\
	\end{minipage}
	\begin{minipage}[t]{0.5\textwidth}
		\textbf{Case 2:} $p \nmid m$
		\begin{flalign*}
		M^{p-1} &\equiv 1\;(\bmod\; p) \text{ (By F$\ell$T)}&\\
		(M^{(p-1)})^{k(q-1)} &\equiv 1^{l(q-1)} \equiv 1\;(\bmod p)\\
		M \cdot M^{k(p-1)(q-1)} &\equiv M\;(\bmod\; p)
		\end{flalign*}
		$R \equiv M \;(\bmod\; p)$ (By Transitivity of Congruences)
	\end{minipage}
	
	In a similar manner, we show $R \equiv M\;(\bmod\; q)$.\\
	
	Since $\gcd(p, q) = 1$, then $R \equiv M\;(\bmod\; pq)$ (CRT)\\
	
	And, since $n=pq$, $R \equiv M\;(\bmod\; n)$\\
	
	As $0 \leq R, M \leq n$, $R=M$\\
	\pagebreak
	
	\underline{\textbf{Why is RSA Secure?}}
	
	Given $(e, n)$ (The public key) and $C$, the ciphertext (or encrypted message), we need to find $d$ to decrypt.\\
	To find $d$, you need $(p-1)(q-1)$, which means factoring $n$. This is a very difficult task for computers, especially with large $n$. This means an eavesdropper has no efficient method of computing $d$ and decoding the message.
\end{document}