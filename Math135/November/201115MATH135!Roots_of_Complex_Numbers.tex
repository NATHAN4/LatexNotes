\documentclass{letter}
\usepackage[margin=0.75in]{geometry}
\usepackage{amsmath}
\usepackage{amssymb}
\usepackage{enumerate}
\usepackage{changepage}
\usepackage{tikz}
\usepackage{pgfplots}
\pgfplotsset{vasymptote/.style={
		before end axis/.append code={
			\draw[densely dashed] ({rel axis cs:0,0} -| {axis cs:#1,0})
			-- ({rel axis cs:0,1} -| {axis cs:#1,0});
		}
	}}

\begin{document}
	\begin{center}
		\LARGE Math135 - November 20, 2015\\
		\large Roots of Exponential Numbers
	\end{center}
	\vspace{0.25 in}
	\underline{\textbf{Complex $\mathbf{n}$'th Roots}}
	
	If $a$ is a complex number and $n$ is a natural number, then the complex numbers that solve $z^n = a$ are called the complex $n$'th roots.
	
	\begin{itemize}
		\item[Ex. ] Find all complex 6'th roots of -64.\\\\
		We need to solve $z^6 = -64$.\\
		Let $z = r(cos \theta + - \sin \theta)$\\
		$r^6 (\cos (6\theta) + i \sin (6\theta)) = 64(\cos \pi + i \sin \pi)$\\\\
		Two complex numbers in polar form are equal iff their moduli are equal and the difference between their arguments is a multiple of $2 \pi$. (Including 0 and negatives)\\\\
		Thus,
		\begin{minipage}[t]{0.5\textwidth}
			\begin{flalign*}
				r^6 &= 64&\\
				4 &= 2
			\end{flalign*}
		\end{minipage}
		\begin{minipage}[t]{0.5\textwidth}
			\begin{flalign*}
				6\theta - \pi &= 2k\pi, k \in \mathbb{Z}&\\
				6\theta &= (2k+1)\pi\\
				\theta &= \frac{\pi}{6} + \frac{\pi}{3}k
			\end{flalign*}
		\end{minipage}\\\\
		
		Note: If we start with $k=0$ and count up, $\theta$ will begin to repeat when $k=6$ since $6(\frac{\pi}{3}) = 2\pi$, and $\sin$ / $\cos$ of $2\pi$ = 0.\\\\
		We get 6 roots with arguments:\\
		\[ \dfrac{\pi}{6}, \dfrac{3\pi}{6}, \dfrac{5\pi}{6}, \dfrac{7\pi}{6}, \dfrac{9\pi}{6}, \dfrac{11\pi}{6} \]\\
		$\therefore$ the roots will be:\\
		\[ \sqrt{3} + 1, \;2i\;, -\sqrt{3} + i, -\sqrt{3} - i, -2i, \sqrt{3} - i \]\\
		
		Notice what happens when we plot these roots on the complex plane.
		\\\\
		
		\begin{center}\begin{tikzpicture}
		\begin{axis}[
		x label style={at={(axis description cs:0.5,-0.1)},anchor=north},
		y label style={at={(axis description cs:-0.1,.5)},rotate=90,anchor=south},
		xlabel={Re},
		ylabel={\;\;Im\;(i)},
		axis equal image,
		axis lines=middle,
		xmin=-2,xmax=2,
		ymin=-2,ymax=2,
		enlargelimits={abs=1cm},
		axis line style={latex-latex},
		yticklabel style={anchor=west},
		ytick={},
		xtick={-1.7, 1.7}
		]

		\addplot[only marks] table {
			1.7 1
			0 2
			-1.7 1
			-1.7 -1
			0 -2
			1.7 -1
		};
		\end{axis}
		\end{tikzpicture}\;\;\;\;\;\;\;\;\;\end{center}
	\end{itemize}
	\clearpage
	
	\underline{\textbf{Complex $\mathbf{n}$'th Roots Theorem (CNRT)}}
	
	Let $n \in \mathbb{N}$. If $r(\cos \theta + i \sin \theta)$ is the polar form of a complex number $a$, then the solutions to $z^n = a$ are:
	\[ \sqrt[n]{r} \bigg( \cos \dfrac{\theta + 2k \pi}{n} + i \sin \dfrac{\theta + 2k \pi}{n} \bigg)  \text{for } k = 0, 1, 2, \dots, n-1\]
	
	Note \begin{enumerate}
		\item Any non-zero complex number will have $n$ distinct $n'th$ roots.
		\item The roots graphed on a complex plane will lie on a circle of radius $r$ centered on the pole (origin) and they will be evenly spaced with angles of $\frac{2\pi}{n}$ between them.
	\end{enumerate}
	
	\underline{\textbf{$\mathbf{n}$'th Roots of Unity}}
	
	Let $n \in \mathbb{N}$ An $n$'th root of unity is a complex number that solves:
	\[ z^n = 1 \]
	
	\begin{itemize}
		\item[Ex. ] Find all 8'th roots of unity.\\\\
		First, note that the complex number $1$ has $r=1$ and $\theta=0$ (Think about where 1 lies on the complex plane)\\\\
		So, we will use CNRT where $n=8, a=1, r=1, \theta = 0$\\
		\begin{flalign*}
			z &= \sqrt[8]{1} \bigg( \cos \dfrac{0 + 2k\pi}{8} + i \sin \dfrac{0 + 2k\pi}{8} \bigg)&\\
			&= \cos \dfrac{k \pi}{4} + i \sin \dfrac{k \pi}{4}, k = 0, 1, \dots, 7\\
			&= 1, \frac{sqrt{2}}{2} +\frac{\sqrt{2}}{2}i, i, -\frac{\sqrt{2}}{2} + \frac{\sqrt{2}}{2}i, -1, -\frac{\sqrt{2}{2}} - \frac{\sqrt{2}}{2}i, -i, \frac{\sqrt{2}}{2} - \frac{\sqrt{2}}{2}i
		\end{flalign*}
		\item[Ex. ] $z^5 = -16\overline{z}$\\\\
		$z = r(\cos \theta + i \sin \theta)$\\\\
		$r^5(\cos (5 \theta) + i \sin(5 \theta)) = 16(\cos \pi + i \sin \pi)(r(\cos (-\theta) + i \sin(-\theta)))$\\
		$r^5(\cos (5 \theta) + i \sin(5 \theta)) = 16r(\cos(\pi - \theta) + i \sin(\pi - \theta))$\\
		
		\begin{minipage}[t]{0.5\textwidth}
			\begin{flalign*}
			r^5 &= 16r&\\
			r^5 - 16r &= 0\\
			r(r^4 - 16) &= 0\\
			r = 0,&\;\; r = 2\\
			\end{flalign*}
			If $r=0$, $z = 0 + 0i$\\
		\end{minipage}
		\begin{minipage}[t]{0.5\textwidth}
			\begin{flalign*}
			5 \theta &= \pi - \theta + 2k\pi&\\
			6 \theta &= \pi + 2k \pi\\
			\theta &= \dfrac{\pi + 2k \pi}{6}\\
			\theta &= \dfrac{\pi}{6} + \dfrac{k\pi}{3}
			\end{flalign*}
		\end{minipage}\\
		$z = \sqrt{3} + i, 2i, -\sqrt{3} + i, -\sqrt{3} - i, -2i, \sqrt{3} - i, 0$
	\end{itemize}
\end{document}