\documentclass{letter}
\usepackage[margin=0.75in]{geometry}
\usepackage{amsmath}
\usepackage{amssymb}
\usepackage{enumerate}
\usepackage{changepage}
\usepackage{tikz}
\usepackage{pgfplots}
\pgfplotsset{compat=1.8}

\pgfplotsset{vasymptote/.style={
		before end axis/.append code={
			\draw[densely dashed] ({rel axis cs:0,0} -| {axis cs:#1,0})
			-- ({rel axis cs:0,1} -| {axis cs:#1,0});
		}
	}}

\begin{document}
	\begin{center}
		\LARGE Math135 - December 02, 2015\\
		\large RQF and RFRP
	\end{center}
	\vspace{0.25 in}
	
	\underline{\textbf{Real Quadratic Factors (RQF)}}\\\\
	Let $f(x)$ be a polynomial with real coefficients. If $c \in \mathbb{C},\; Im(c) \neq 0$ is a root of $f(x)$, then there exists a real quadratic factor of $f(x)$ with $c$ as a root.\\\\
	Proof:
	\begin{itemize}
		\item[ ] Let $f(x)$ be a polynomial with real coefficients, let $c \in \mathbb{C}$ be a root of $f$ where $Im(c) \neq 0$.\\\\
		By CJRT, $\overline{c}$ is also a root.\\\\
		So, $(x-c)(x-\overline{c})$ is a factor of $f(x)$. We know these are different factors since the imaginary parts $\neq$ 0.
		\begin{flalign*}
			&((x-c)(x-\overline{c})&\\
			&= x^2 - cx - \overline{c}x + c\overline{c}\\
			&= x^2 - (c-\overline{c})x + c\overline{c}\\
			&= x^2 - 2Re(c)x + |c|^2
		\end{flalign*}
		$2Re(c) \in \mathbb{R},\;\;\; |c|^2 \in \mathbb{R}$, so we have a real quadratic factor with $c$ as a root, QED.
		
		\item[Ex. ] $f(x) = x^4 - 5x^3 + 16x^2 - 9x - 13$\\
		$= (x-2 + 3i)(x-2-3i)(x - \frac12 + \frac{\sqrt5}{2})(x - \frac12 - \frac{\sqrt5}{2})$\\
		$= (x^2 - 4x + 13)(x - \frac12 + \frac{\sqrt 5}{2})(x - \frac12 - \frac{\sqrt5}{2})$\\
		\item[Ex. ] $f(x) = x^4 + 2x^2 + 1$\\
		$= (x^2)^2 + 2x^2 + 1$\\
		$= (x^2+1)^2$\\
		$= (x-i)^2 (x+i)^2$
	\end{itemize}
	\underline{\textbf{Real Quadratic Factors (RQF)}}\\\\
	Let $f(x)$ be a polynomial with real coefficients. Then, $f(x)$ can be written as a product of real linear and real quadratic factors.\\\\
	Proof:\\
	\begin{itemize}
		\item[ ] Let $f(x)$ be a polynomial of degree $n$ with real coefficients.\\
		Consider factoring $f(x)$ over $\mathbb{C}\left[ x \right]$. We will get $n$ complex roots, and be able to express $f(x)$ as a product of $n$ linear factors.\\ 
		Some of these factors may be real, and some of them may be non-real.\\
		Since $f$ has real coefficients, any imaginary roots come in pairs. (CJRT) Multiplying these pairs of imaginary roots together will produce a quadratic with real coefficients. So, any real polynomial can be broken down into linear and quadratic terms.\\\\
	\end{itemize}
		This is the last of the material that will be covered on the final exam. :D
	
	
\end{document}