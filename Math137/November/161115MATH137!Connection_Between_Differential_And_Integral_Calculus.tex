\documentclass{letter}
\usepackage[margin=0.75in]{geometry}
\usepackage{amsmath}
\usepackage{amssymb}
\usepackage{enumerate}
\usepackage{changepage}

\begin{document}
	\begin{center}
		\LARGE Math137 - November 16, 2015\\
		\large 
	\end{center}
	\vspace{0.25 in}
	\underline{\textbf{Connection between Differential and Integral Calculus}}
	
	Consider an arbitrary $f(x)$ which is continuous on the closed interval $[a, b]$.
	
	The area function $A(x)$  is defined to be the area under the curve from the point $a$ to an arbitrary point $x$ in the interval $[a, b]$.
	
	Consider: \[ A'(x) = \lim_{h \to 0} \frac{A(x + h) - A(x)}{h} \]
	
	Then, for $h>0$, $A(x+h) -A(x)$ represents the difference of two areas. This can be approximated by the area of a rectangle with base $h$ and height $f(c)$ where $c$ is some point in the interval $[x, x+h]$.
	
	Thus: $\dfrac{A(x+h) - A(x)}{h} \approx \dfrac{h f(c)}{h} = f(c)$
	
	In the limit, as $h \to 0$, we obtain:
	
	\[ A'(x) = \lim_{h \to 0} \dfrac{A(x+h) - A(x)}{h} = f(c) \]
	
	\underline{\textbf{Computing Areas From First Principles}}
	
	\begin{itemize}
		\item[\textbf{Ex. }] Find the area $A$ lying under the line $y = x + 1$ above the $x$ axis and between $x=0$ and $x=2$.
		
		By inspection, the area is equal to $\frac{1}{2}(2)(1+3) = 4$
		
		We verify this result as follows:
		
		Divide $[0, 2]$ into $n$ subintervals, each of length $\frac{2}{n}$.
		
		Take height of each rectangle to be that of the right endpoint.
		
		Area of the $i$'th rectangle $A_i = f(x_i)\Delta x_i$ where $f(x_i) = x_i + 1 = 1 + i \frac{2}{n}$, $\Delta x_i = \Delta x = \frac{2}{n}$.
		
		Then,
		\begin{flalign*}
			A &= \lim_{n \to \infty} \sum_{i=1}^{n} A_i = \lim_{n \to \infty} \sum_{i=1}^{n} \frac{2}{n}(1 + i\frac{2}{n})\\ 
			A &= \lim_{n \to \infty} \bigg[ \sum_{i=1}^{n} \frac{2}{n} + i \frac{4}{n^2} \bigg] = \lim_{n \to \infty} \bigg[ \frac{2}{n} \sum_{i=1}^{n} 1 + \frac{4}{n^2} \sum_{i=1}^{n} i \bigg]\\
			A &= \lim_{n \to \infty} \bigg[ 2 + \frac{4}{n^2} (\frac{n^2 + n}{2})\bigg] = 2 + \lim_{n \to \infty} \bigg[ 2(1 + \frac{1}{n})\bigg]\\
			A &= 2 + 2 = 4 
		\end{flalign*}
		
		Wow.
		
		\underline{\textbf{Summation Formulas}}
		\begin{enumerate}[a)]
			\item $ \displaystyle \sum_{i=1}^n 1 = 1 + 1 + \dots + 1 \ n$
			\item $ \displaystyle \sum_{i=1}^n i = 1 + 2 + 3 + \dots + n = \frac{n(n+1)}{2}$
			\item $ \displaystyle \sum_{i=1}^{n} i^2 = 1^2 + 2^2 + \dots + n^2 = \frac{1}{6} (n)(n+1)(2n + 1)$
			\item $ \displaystyle \sum{i=1}{n} i^3 = 1^3 + 2^3 + \dots + n^3 = (\frac{n}{2}(n+1))^2$
			\item $\sum_{i=1}^n r^{i-1} = 1 + r + r^2 + \dots + r^{n-1} = \dfrac{r^n - 1}{r-1}$
		\end{enumerate}
	\end{itemize}
\end{document}