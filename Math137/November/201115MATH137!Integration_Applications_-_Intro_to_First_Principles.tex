\documentclass{letter}
\usepackage[margin=0.75in]{geometry}
\usepackage{amsmath}
\usepackage{amssymb}
\usepackage{enumerate}
\usepackage{changepage}

\begin{document}
	\begin{center}
		\LARGE Math137 - November 20, 2015\\
		\large Integration Applications - Intro to First Principles
	\end{center}
	\vspace{0.25 in}
	\underline{\textbf{Integral Applications:}}
	
	\textbf{Note for non-physics people:} An object falling close to the earths surface has upward acceleration of -9.8$m/s^2$.
	
	\begin{itemize}
		\item[Ex. ] A stone is dropped from the Space Deck of the CN Tower 450m above the ground.
		\begin{enumerate}[a)]
			\item Find the height $h$ of the stone above the ground at time $t$.\\\\
			We know $a(t) = -9.8$\\
			$v(t) = -9.8t + c$\\
			The stone is dropped, so its initial velocity is 0.\\\\
			$v(0) = 0 =  -9.8(0) + c$\\
			$\therefore c = 0$\\
			$\therefore v(t) = -9.8(t)$\\\\
			$h(t) = \dfrac{-9.8t^2}{2} + c_1$\\\\
			The stone was dropped at $h=450$.\\
			$h(0) = 450 \implies \dfrac{-9.8(0)}{2} + c_1 = 450$\\
			$c_1 = 0$\\
			$\therefore h(t) = -4.9t^2 + 450$
			\item How long does it take the stone to hit the ground?\\\\
			We need to find t when $h(t) = 0$\\
			\begin{flalign*}
				0 &= -4.9t^2 + 450&\\
				t^2 &= \dfrac{450}{4.9}\\
				t &= +\sqrt{\dfrac{450}{4.9}}\;\;\;\text{(Time cannot be negative in this application)}\\
				t &\approx 9.58s
			\end{flalign*}
			\item What is the stones impact velocity?\\\\
			We need to find $v(9.58)$\\\\
			\begin{flalign*}
				v(t) &= -9.8t&\\
				v(9.58) &= -9.8(9.58)\\
				&= -93.9 m/s
			\end{flalign*}
		\end{enumerate}
	\end{itemize}
	\clearpage
	\underline{\textbf{First Principles:}}
	
	\textbf{Note: } in this section I reference 'Area Under a Curve', a pdf available under Mike Eden's section on Learn.
	
	\begin{enumerate}
		\item \textbf{What would be another way to estimate the total distance Caroline travelled?}\\
		We could have used the right endpoint of each interval instead of the left. (Denoted $R_6$).
		\item \textbf{How could we make the estimate better?}\\
			We could have use the midpoint speed of each interval instead of left or right.\\
			We could decrease the width of our intervals (i.e. increase the number of intervals)
		\item \textbf{How is this related to area under a curve?}\\
		The antiderivative of velocity is distance.\\
		The area under the velocity curve gives distance over an interval $[t_1, t_2]$.\\\\
		Conclusion: The anti-derivative of velocity is the area under the velocity curve.
	\end{enumerate}
\end{document}